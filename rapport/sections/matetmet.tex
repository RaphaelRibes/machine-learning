On regarde le nombre de \texttt{\#}, \texttt{@} et lien par tweet selon le type de tweet scientifique (\autoref{fig:hashtag_count}).

\begin{figure}[H]
    \centering
    \includegraphics[width=1\textwidth]{images/hashtag_links_mentions_count_outliers}
    \caption{Nombre de hashtags par tweet selon le type de tweet scientifique.}
    \label{fig:hashtag_count}
\end{figure}

Après avoir réalisé des tests de student sur chaque variable, les \textit{p-values} ne descendent pas en dessous de 0.18 donc on ne remarque pas de différence significative entre les tweets scientifiques et non scientifiques.
On peut ainsi conclure que ces variables ne sont pas pertinentes pour la classification des tweets scientifiques, on les retirera de notre dataset.