Cette étude a permis de comparer plusieurs modèles classiques de classification appliqués à des tweets selon leur rapport à la science.
Malgré les défis liés au bruit, à l’ambiguïté sémantique et au déséquilibre partiel des classes, les performances obtenues sont encourageantes.
Le modèle SVM s’est révélé le plus performant dans la majorité des tâches.
Mais plusieurs pistes d’amélioration peuvent être envisagées pour les travaux futurs.
Tout d’abord, l’intégration de modèles de langage pré-entraînés comme BERT ou CamemBERT pourrait permettre une meilleure prise en compte du contexte sémantique et améliorer la performance, notamment sur les classes minoritaires ou ambigües.
De plus, l’exploration de techniques de data augmentation textuelle ou de rééquilibrage plus avancé pour le multi-label (comme les algorithmes de suréchantillonnage spécifiques au texte) pourrait renforcer la robustesse des modèles face au déséquilibre des combinaisons de labels.
Enfin, une analyse plus fine des erreurs de classification, couplée à une révision éventuelle des annotations, permettrait d’améliorer la qualité du jeu de données, ce qui constitue une condition essentielle pour des modèles plus précis et interprétables.

