\subsection{Limites des Modèles}\label{subsec:limites-des-modeles}
Malgré des performances globalement satisfaisantes, chaque modèle présente certaines limites :
\begin{itemize}
  \item \textbf{Modèles linéaires (régression logistique, SVM) :} sensibles aux features bruitées ou non pertinentes, ils peuvent peiner à capturer des relations complexes dans les données textuelles.
  \item \textbf{Naive Bayes :} basé sur l’indépendance conditionnelle des features, il tend à sursimplifier les relations entre les mots et peut produire des résultats biaisés dans le cas de dépendances lexicales fortes.
  \item \textbf{k-NN :} sensible à la dimensionnalité et aux données bruitées, ce modèle peut être affecté par des mots non informatifs ou rares présents dans les tweets.
  \item \textbf{Random Forest :} bien qu’efficace, il peut sur-apprendre certains patterns lorsque le déséquilibre des classes n’est pas bien traité.
\end{itemize}

\subsection{Analyse des Features Importantes}\label{subsec:analyse-des-features-importantes}
Afin de mieux comprendre le comportement des modèles, nous avons extrait les features les plus importantes à l’aide du classifieur Random Forest.
Cette démarche nous a permis d’identifier les mots les plus discriminants entre les tweets scientifiques et non-scientifiques.
Ces termes ont ensuite été analysés en fonction de leur fréquence d’apparition, représentée sous forme de bar plot (cf Figure X), ce qui a facilité l’interprétation des décisions du modèle et la compréhension des signaux textuels les plus informatifs.

\subsection{Inspection Manuelle des Erreurs}\label{subsec:inspection-manuelle-des-erreurs}
Pour les cas où les performances des modèles étaient jugées insuffisantes, une analyse manuelle des erreurs de classification a été réalisée.
Nous avons examiné les tweets mal classés, en particulier les faux positifs et faux négatifs, afin d’identifier des exemples ambigus ou potentiellement mal annotés.
Cette inspection a permis de mieux comprendre les limites des modèles face à des contenus complexes ou \textit{borderlines}.
Lorsque des motifs d’erreur récurrents ont été repérés, nous avons tenté d’ajuster le prétraitement des données ou d’équilibrer manuellement certaines combinaisons de labels.
Enfin, des modifications ciblées du pipeline ont été mises en place pour corriger ces erreurs, tout en veillant à ne pas nuire à la capacité de généralisation des modèles sur des données nouvelles.
